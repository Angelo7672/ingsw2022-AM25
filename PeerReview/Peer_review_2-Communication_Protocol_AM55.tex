\documentclass[12pt]{article}
\usepackage[utf8]{inputenc}
\usepackage[T1]{fontenc}
\usepackage[italian]{babel}

\title{Peer-Review 2: Protocollo di Comunicazione}
\author{Chiara Di Pasquale, Angelo Attivissimo, Ginevra Bozza\\Gruppo 25}

\begin{document}

\maketitle

Valutazione del protocollo di comunicazione del gruppo 55.

\section{Lati positivi}
\begin{itemize}

\item L'aver segnato i loop nei sequence diagrams
\item Il messaggio che dice al client, in fase di login, quanti giocatori mancano dopo essersi collegato 
\item L'avere già pensato allo stato del client attivo/inattivo
\end{itemize}

\section{Lati negativi}
\begin{itemize}
\item Il fatto che, in fase di login, non venga chiesto il Wizard
\item Non c'è l'inizializzazione dell'utente. Ad esempio, non viene mandata al client la squadra
\item Nel diagramma dell'action phase c'è un piccolo refuso: avete scritto "cloud" al posto della mensa (punto 7)
\item Nell'action phase, forse è meglio usare un unico messaggio per muovere gli studenti, invece che suddividerlo in tre messaggi diversi
\item Vi consigliamo di inserire un messaggio dopo Choose cloud, che il server manda al client successivo per svegliarlo dalla sua fase di attesa
\end{itemize}

\section{Confronto tra le architetture}
\begin{itemize}
\item Noi non abbiamo pensato ad un messaggio che arriva dal server quando ci si collega ad una partita già iniziata, ma lo metteremo e faremo si che ne venga mandato uno nuovo ogni volta che si collega un nuovo client.
\item Non abbiamo pensato allo status del client (attivo/inattivo), ma verrà aggiunto anche nel nostro progetto.
\item Non abbiamo ben evidenziato i loop nel nostro sequence diagram, li renderemo più evidenti con delle notazioni come avete fatto voi
\end{itemize}
\end{document}
